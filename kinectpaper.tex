\listfiles
\documentclass[manuscript, review, screen]{acmart}
\setcitestyle{super,sort&compress}
\citestyle{acmauthoryear}
\usepackage{graphicx}
\usepackage{amsmath, amsthm, amssymb}
\usepackage{amsfonts}
\usepackage{tabularx}
\usepackage{multirow}
\usepackage{booktabs}
\usepackage[printonlyused]{acronym}
\usepackage{paralist}
\usepackage{enumitem}
\usepackage{subcaption} 
\usepackage[ruled]{algorithm2e}
%\usepackage{algorithmic}

\setlist{nolistsep}

\newacro{PGCE}{Postgraduate Certificate of Education}
\newacro{UTToF}{User Tracking Time-of-Flight}

\newcommand{\tickYes}{\checkmark}
\newcommand{\crossNo}{$\times$}

% Metadata Information
% \acmJournal{TOCHI}
% \acmVolume{9}
% \acmNumber{4}
% \acmArticle{39}
% \acmYear{2018}
% \acmMonth{3}

% Copyright
%\setcopyright{acmcopyright}
\setcopyright{acmlicensed}
%\setcopyright{rightsretained}
%\setcopyright{usgov}
%\setcopyright{usgovmixed}
%\setcopyright{cagov}
%\setcopyright{cagovmixed}

% DOI
%\acmDOI{}

% Document starts
\begin{document}
% Title portion
\title{Controlling Classroom Technology with Upper-Body Gestures
  {\emph{and}} Improving Upper-Body Gesture Controls for Teacher Use}

\author{James McNaughton}
%\orcid{}
\affiliation{%
  \institution{Durham University}
  \streetaddress{South Road}
  \city{Durham}
  \postcode{DH1 3LE}
  \country{UK}}
\email{j.a.mcnaughton@durham.ac.uk}
\author{Tom Crick}
\orcid{0000-0001-5196-9389}
\affiliation{%
  \institution{Swansea University}
  \streetaddress{Singleton Campus}
  \city{Swansea}
  \postcode{SA2 8PP}
  \country{UK}}
\email{thomas.crick@swansea.ac.uk}

\renewcommand\shortauthors{McNaughton, J. and Crick, T.}

% TODO Add Liz Burd as co-author

\begin{abstract}

% TODO New abstract to explain the 2 phase approach to the study (with the gesture gathering exercise before hand).

\end{abstract}

\keywords{Kinect, gestures, education, multi-touch, classroom technology}

\maketitle

\section{Introduction}
\label{sec:intro}

The uses of technology in the classroom are
growing~\cite{Schrum2008,Lloyd2011,Robertson2012,mcnaughton-et-al:jce2017},
further stimulated by significant reforms of digital skills and computer science education in various nations, especially across the UK~\cite{brown-et-al:toce2014}.
With this growth, the need for teachers to be able to control the deployed technologies increases~\cite{Apple1990,Selwyn2010,Selwyn2011}.
Without the ability to influence or control classroom technology, teachers may be unable to manage learning interaction or intervene when students start to lose focus on their current task~\cite{Chen2005,Karabenick2011}.

Many current systems that allow teachers to control technology in the classroom require the use of a teacher-centric interface~\cite{Dagdag2011,Kuhn2005,Vila,Zhou2010}.
Whether static, where the interface remains stationary during its use, or mobile, where the interface can be carried to new locations during its use, these interfaces require the teacher to momentarily take their attention away from the students.
This division of attention caused by the distraction of an interface could have a detrimental effect on the quality of a teacher's interaction with their students.
The disruption in communication between the students and the teacher this causes can be undesirable in many circumstances.
Therefore, a method of controlling technology in the classroom without breaking this interaction would be beneficial.
One such possible method is to make use of physical gestures, where a user performs an action which is identified by a monitoring system, to issue commands to technology in the classroom.

The use of gestures, rather than a more standard interface, could allow teachers to issue commands in a more effective manner.
Time is saved by not requiring the teacher to travel to their control interface.
Even when there is no travel time, such as when mobile interfaces are used, gestures have the potential to be executed quicker than alternative input and control methods~\cite{Dulberg1999,Moyle2001}.
Quicker execution of the commands should afford the teacher more time to observe and aid students.
In addition, physical gestures should be less intrusive on the interaction between students and the teacher. 
Many interfaces, specifically touch-screen mobile devices such as tablets, do not facilitate eyes-free interaction~\cite{Brewster2003}.
This means that teachers using a static or mobile interface which utilises a visual output are required to dedicate a portion of their attention to its use.
This division of attention can interrupt interaction between teachers and students.
The use of physical gestures should allow teachers to continue interacting with students while issuing commands to a classroom technology's control system.

Teachers in technology enhanced classroom also acquire additional administration responsibilities~\cite{Kuhn2005} such as managing the consequence of faults with the devices used.
A physical gesture interface may reduce the overheads of such additional responsibilities by allowing teachers to quickly execute administrative tasks from any location in the classroom.

The potential benefits of physical gestures make its implementation into a classroom software framework desirable.
For a system such as this, a series of gestures must be produced which can be used to execute the various control commands within the framework.

% TODO Update Summary of Structure .

% The remainder of this paper is as follows. 
% Section~\ref{sec:related1} discusses gesture detection technologies, specifically time-of-flight cameras.
% Section~\ref{sec:classcontrol1} considers how the time-of-flight cameras could be used to adapt a classroom-based technology to utilise physical gestures.
% What constitutes an effective gesture is discussed in Section~\ref{sec:gestures1}.
% A focus group based study is outlined in Section~\ref{sec:focusgroup1} which was carried out to identify intuitive gestures for classroom control commands.
% The results of the study are outlined in Section~\ref{sec:focusgroupresults1} and their implications are discussed in Section~\ref{sec:discussion1}. 
% Implementation of a user generated set of gestures from observations on the focus group is detailed in Section~\ref{sec:implementation1}.
% Section~\ref{sec:evaluation1} discusses a study using the implemented physical gesture control system.
% Section~\ref{sec:evaluationresults1} presents the results of this study and Section~\ref{sec:evaluationdiscussion1} elaborates on the implications of the results.
% Future developments utilising findings from the study and our conclusions are presented in Section~\ref{sec:conclusions1}.

\section{Background} 
\label{sec:background}

% TODO Needs more recent content.

% TODO Double check that the Kinect actually is a time-of-flight and UTToF camera

\section{Classroom Technology Orchestration} 
\label{sec:orchestration}

% TODO RQ: Whats a better control mechanism for teachers: smart boards, tablets or the Kinect?
% Implementation

% TODO Detail full developments (i.e. support for multiple Kinects and different command issuing trees)

\section{Study Design} 
\label{sec:studydesign}

% TODO Highlight need for a suitable gesture set to be generated

% TODO Highlight need for an iterative design due to exploring alternative implementations, therefore there are multiple phases to the study.

\subsection{Gathering User Generated Gestures} 
\label{sec:gatheringgestures}

% TODO Consolidate all information on the user gesture gathering (approach and results).


\section{Study Phase 1} 
\label{sec:studyphase1}

% TODO Study 1 design (single Kinect with the original gesture tree).

% TODO Results

% TODO Observations (Note how the Kinect's poor range and tracking of movement were hugely problematic and needed to be corrected for any further studies.)


\section{Study Phase 2} 
\label{sec:studyphase2}

% TODO Two Kinects with an improved gesture tree intended to counter the Kinect's limitations.).

% TODO Results

% TODO Observations

\section{Conclusions}
\label{sec:conclusions}

% TODO Summary of the whole thing.


\begin{acks}

This work was partially funded under the UK's EPSRC/ERSC Teaching and Learning Research Programme (TLRP) {\emph{SynergyNet}} project (RES-139-25-0400).

% TODO Include thanks to Andrew J-G

\end{acks}

% Bibliography
\bibliographystyle{ACM-Reference-Format}
\bibliography{kinectpaper}


\end{document}
